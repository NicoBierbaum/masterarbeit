\UseRawInputEncoding
\setcounter{secnumdepth}{3}			% 3 Ebenen bei der Kapitelnumerierung

%% nützliche Packet -------------------------------------------------------------------------------
\usepackage[ngerman]{babel}			% neue deutsche Rechtschreibung
\usepackage{scrhack}
\usepackage{placeins}
\usepackage{microtype}
%\usepackage{lmodern}
\usepackage{subcaption}

\usepackage{cmbright}
\usepackage[T1]{fontenc}			%\usepackage[OT1]{fontenc} alt
\usepackage{blindtext}				% Mit \blindtext wird Dummytext eingefügt
\usepackage{mdwlist}				% kompaktere Listen mit itemize\cdot und co
\usepackage{graphicx}				% Grafiken
\usepackage{epstopdf}               % epsBilder automatisch in pdfBilder konvertieren
\usepackage{ltxtable}				% longtable und tabularx Pakete (XSpalten)
\usepackage{multirow}				% Mehrzeiliges in Tabellen
\usepackage[dvipsnames,table,xcdraw]{xcolor}			% alternierende Farben in Tabellen
\usepackage{rotating}				% Rotieren von Text & co
%\usepackage{color}			    	% Alles in Bunt und Farbe
\usepackage{multicol}				% multicol halt
\usepackage{amsmath}				% für mathematische Formeln
\usepackage{amsthm}				    % für theoremstyle
\usepackage{mathtools}
\usepackage{amssymb}				% für mathematische Symbole
\usepackage{wrapfig}				% umflossene figures
\usepackage{float}				    % für [h!] bzw. [H] positionierung
\usepackage{booktabs}	            % Für Excel2Latex Tabellen
\usepackage{eurosym}				% Eurosymbol verwenden
\usepackage{url}				    % Weblinks einbinden  z.B. http://www.fh-muenster.de
\usepackage{etoolbox}
\appto\UrlBreaks{\do\a\do\b\do\c\do\d\do\e\do\f\do\g\do\h\do\i\do\j
	\do\k\do\l\do\m\do\n\do\o\do\p\do\q\do\r\do\s\do\t\do\u\do\v\do\w
	\do\x\do\y\do\z}
\usepackage{fancybox}				% für Kästen und Boxen
\usepackage{caption}                % Erweiterte Einstellung für Beschriftungen
\usepackage{array}                  % erweiterte Optionen
\usepackage{icomma}                 % verringert den Abstabd hinter dem Komma bei Kommawerten
\usepackage{wrapfig}				% Textumflossene Bilder innerhalb des Dokumentes
\usepackage{setspace}				% Zeilenabstand definieren

%% PDF Optionen (Verweise und co) -----------------------------------------------------------------
\usepackage{ifxetex}
\ifxetex
\usepackage[babel, german=quotes]{csquotes}
\usepackage{xltxtra}
\usepackage{etex}
\usepackage[dvipsnames]{pstricks}
\usepackage{pst-plot}
\usepackage{pstricks-add}
\usepackage{auto-pst-pdf}
\usepackage[pdfa, xetex]{hyperref}
\else
\usepackage[utf8]{inputenc}			% UTF-8 encoding
\usepackage[babel, german=quotes]{csquotes}
\usepackage[pdftex, raiselinks, pdfpagelabels, hypertexnames=false, pdfstartview={Fit}]{hyperref}
\definecolor{lightblue}{rgb}{0.9,0.9,1}
\fi
%% Angaben für die PDF-Datei
\hypersetup{
	pdfauthor = {Max Mustermann},
	pdftitle =  {Businessplan fiktives Unternehmen ShotBot},
	pdfsubject = {Projektarbeit},
 	pdfkeywords = {Keyword1, Keyword2, ...},
 	pdfcreator = {LaTeX with hyperref package}
}
%\setlength\parindent{0pt}
%% Kopf- und Fußzeile -----------------------------------------------------------------------------
\usepackage[headsepline]{scrlayer-scrpage}
\pagestyle{scrheadings}
\clearpairofpagestyles
%\rehead{\pagemark}
%\lohead{\pagemark}
\ihead{\headmark} % Neudefinition
\ohead{\pagemark}
%\automark[chapter]{section} % Kopfzeile: [rechts]{links}

%% Seitenränder -----------------------------------------------------------------------------------
\usepackage[a4paper,inner=4cm,outer=2.5cm,top=2cm,bottom=2cm,includeheadfoot]{geometry}

%% Literaturverzeichnis ---------------------------------------------------------------------------
\usepackage[style=numeric-comp,sorting=none,backend=biber]{biblatex}

\DeclareFieldFormat{labelalpha}{\textsc{#1}}    % Setzt den Autor in Kapitälchen 
\renewcommand*{\mkbibnamefamily}[1]{\textsc{#1}}  % Setzt den Nachnamen in Kapitälchen 
\renewcommand*{\mkbibnamegiven}[1]{\textsc{#1}}  % Setzt den Vornamen in Kapitälchen 
\renewcommand*{\mkbibnameprefix}[1]{\textsc{#1}}  % Setzt den Nachnamen in Kapitälchen 
\DeclareFieldFormat{title}{\glqq{#1}\grqq}      % Setzt den Titel in Anführungszeichen
\DeclareFieldFormat{date}{({#1})}               % Setzt das Jahr in Klammern
\makeatletter
\def\blx@maxline{77}
\makeatother
\addbibresource{LiteraturDB.bib} % Bibliographie laden

%% Abkürzungsverzeichnis: -------------------------------------------------------------------------
\usepackage{nomencl}
% Deutsche Überschrift
\renewcommand{\nomname}{Glossar und Abkürzungsverzeichnis}
% Punkte zwischen Abkürzung und Erklärung
\setlength{\nomlabelwidth}{.20\hsize}
\renewcommand{\nomlabel}[1]{#1 \dotfill}
% Zeilenabstände verkleinern
%\setlength{\nomitemsep}{-\parsep}
\makenomenclature
%% ---------------------------------------------------------------------------------------------------
%% Abkürzungsverzeichnis erstellen: ------------------------------------------------------------------
%% 1. LATEX Lauf durchführen
%% 2. Ein DOS-Fenster (Eingabeaufforderung) öffen und in den Ordner mit der Datei "thesis.tex" wechseln.
%% 3. Den Befehl   makeindex thesis.nlo -s nomencl.ist -o thesis.nls   ausführen
%% 4. LATEX Lauf durchführen. Das Abkürzungsverzeichnis sollte jetzt vorhanden sein.

%% Quelltexte -------------------------------------------------------------------------------------
\usepackage{listings}
\lstset{ %
	basicstyle=\scriptsize, 		% the size of the fonts that are used for the code
	numbers=left,				% where to put the line-numbers
	numberstyle=\footnotesize,		% the size of the fonts that are used for the line-numbers
	backgroundcolor=\color{white},		% choose the background color. You must add \usepackage{color}
	commentstyle=\color{green},
	stringstyle =\color{orange},
	xleftmargin=1.5em,
	xrightmargin=1em,
	frame=tb,				% adds a frame around the code
	tabsize=2,				% sets default tabsize to 2 spaces
	captionpos=b,				% sets the caption-position to top or bottom
	breaklines=true,			% sets automatic line breaking
	breakatwhitespace=false			% sets if automatic breaks should only happen at whitespace
	literate =%
	 {#}{\#}1
}
\renewcommand*{\lstlistingname}{Quelltext}
\renewcommand*{\lstlistlistingname}{Quelltextverzeichnis}

%% Tikz für (komplizierte) bunte Bildchen ---------------------------------------------------------
\usepackage{tikz}
\usetikzlibrary{positioning,mindmap,shapes,shapes.multipart,shadows,arrows,patterns,topaths}
\usepackage{pgfplots}
\pgfplotsset{width=7cm,compat=newest}

\definecolor{gelb}{HTML}{FFFCCC}
\definecolor{gruen}{HTML}{CFFFCC}
\definecolor{blau}{HTML}{CCE9FF}
\definecolor{dunkelblau}{HTML}{99b7EE}
\definecolor{rot}{HTML}{FFA8A8}
\definecolor{orange}{HTML}{FFD28F}

\lstset
{
	breaklines=true,
	tabsize=3,
	showstringspaces=false
}


\lstdefinestyle{Common}
{
	extendedchars=\true,
	language={[Visual]Basic},
	frame=single,
	%===========================================================
	framesep=3pt,%expand outward.
	framerule=0.4pt,%expand outward.
	xleftmargin=3.4pt,%make the frame fits in the text area. 
	xrightmargin=3.4pt,%make the frame fits in the text area.
	%=========================================================== 
	rulecolor=\color{Red}
}

\lstdefinestyle{A}
{
	style=Common,
	backgroundcolor=\color{Yellow!10},
	basicstyle=\scriptsize\color{Black}\ttfamily,
	keywordstyle=\color{Orange},
	identifierstyle=\color{Cyan},
	stringstyle=\color{Red},
	commentstyle=\color{Green}
}

\lstdefinestyle{B}
{
	style=Common,
	backgroundcolor=\color{Black},
	basicstyle=\scriptsize\color{White}\ttfamily,
	keywordstyle=\color{Orange},
	identifierstyle=\color{Cyan},
	stringstyle=\color{Red},
	commentstyle=\color{Green}
}
