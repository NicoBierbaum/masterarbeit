\UseRawInputEncoding
\setcounter{secnumdepth}{3}         % 3 levels in section numbering

%% Useful packages -------------------------------------------------------------------------------
\usepackage[english]{babel}         % English language
\usepackage{scrhack}
\usepackage{placeins}
\usepackage{subcaption}
\usepackage{lmodern}
\usepackage{cmbright}
\usepackage{enumitem}               % Package for customized lists
\usepackage[OT1]{fontenc}
\usepackage{blindtext}              % Use \blindtext to insert dummy text
\usepackage{mdwlist}                % More compact lists with itemize and co
\usepackage{graphicx}               % Graphics
\usepackage{epstopdf}               % Automatically convert eps images to pdf
\usepackage{ltxtable}               % Combines longtable and tabularx
\usepackage{multirow}               % Multi-row cells in tables
\usepackage[dvipsnames,table,xcdraw]{xcolor}  % Alternating colors in tables
\usepackage{rotating}               % Rotate text and other elements
\usepackage{multicol}               % Multicolumn formatting
\usepackage{amsmath}                % Mathematical formulas
\usepackage{amsthm}                 % Theorem style
\usepackage{mathtools}
\usepackage{amssymb}                % Mathematical symbols
\usepackage{wrapfig}                % Wrapping figures
\usepackage{float}                  % For [h!] or [H] positioning
\usepackage{booktabs}               % For tables from Excel to LaTeX
\usepackage{eurosym}                % Use Euro symbol
\usepackage{url}                    % URLs
\usepackage{etoolbox}
\appto\UrlBreaks{\do\a\do\b\do\c\do\d\do\e\do\f\do\g\do\h\do\i\do\j
    \do\k\do\l\do\m\do\n\do\o\do\p\do\q\do\r\do\s\do\t\do\u\do\v\do\w
    \do\x\do\y\do\z}
\usepackage{adjustbox}
\usepackage{fancybox}               % Boxes and frames
\usepackage{caption}                % Extended settings for captions
\usepackage{array}                  % Extended options for arrays and tables
\usepackage{icomma}                 % Reduces space after comma in decimal numbers
\usepackage{setspace}               % Define line spacing

%% PDF options (links and more) -----------------------------------------------------------------
\usepackage{pdfpages}
\usepackage[utf8]{inputenc}         % UTF-8 encoding
\usepackage[english=british]{csquotes}
\usepackage[pdftex, raiselinks, pdfpagelabels, hypertexnames=false, pdfstartview={Fit}]{hyperref}
\definecolor{lightblue}{rgb}{0.9,0.9,1}

%% PDF metadata
\hypersetup{
    pdfauthor = {Nico Bierbaum},
    pdftitle =  {Scientific Work},
    pdfsubject = {Density and Shrinkage Evaluation of 316L Printed Parts in the FDM Process},
    pdfkeywords = {3D printing, filament, metal, density, tensile test, shrinkage},
    pdfcreator = {LaTeX with hyperref package}
}

%% Header and footer -----------------------------------------------------------------------------
\usepackage[headsepline]{scrlayer-scrpage}
\pagestyle{scrheadings}
\clearpairofpagestyles
\ihead{\headmark} % Redefine
\ohead{\pagemark}

%% Page margins -----------------------------------------------------------------------------------
\usepackage[a4paper,inner=4cm,outer=2.5cm,top=2cm,bottom=2cm,includeheadfoot]{geometry}

%% Bibliography -----------------------------------------------------------------------------------
\usepackage[style=numeric-comp,sorting=none,backend=biber]{biblatex}

\DeclareFieldFormat{labelalpha}{\textsc{#1}}    % Sets the author in small caps 
\renewcommand*{\mkbibnamefamily}[1]{\textsc{#1}}  % Sets the last name in small caps 
\renewcommand*{\mkbibnamegiven}[1]{\textsc{#1}}  % Sets the first name in small caps 
\renewcommand*{\mkbibnameprefix}[1]{\textsc{#1}}  % Sets the prefix in small caps 
\DeclareFieldFormat{title}{``#1''}      % Sets the title in quotation marks
\DeclareFieldFormat{date}{({#1})}               % Sets the date in parentheses
\makeatletter
\def\blx@maxline{77}
\makeatother
\addbibresource{LiteratureDB.bib} % Load bibliography

%% Glossary and abbreviations ---------------------------------------------------------------------
\usepackage{nomencl}
% English title
\renewcommand{\nomname}{Glossary and List of Abbreviations}
% Dots between abbreviation and explanation
\setlength{\nomlabelwidth}{.20\hsize}
\renewcommand{\nomlabel}[1]{#1 \dotfill}
% Reduce line spacing
%\setlength{\nomitemsep}{-\parsep}
\makenomenclature
\setlength{\headsep}{1.5\baselineskip}

%% ---------------------------------------------------------------------------------------------------
%% Create glossary and abbreviations: ------------------------------------------------------------------
%% 1. Run LATEX
%% 2. Open a command prompt and navigate to the folder with the "thesis.tex" file.
%% 3. Run the command   makeindex thesis.nlo -s nomencl.ist -o thesis.nls
%% 4. Run LATEX again. The glossary and abbreviations should now be present.

%% Source code -------------------------------------------------------------------------------------
\usepackage{listings}
\lstset{ %
    basicstyle=\scriptsize,         % the size of the fonts that are used for the code
    numbers=left,                   % where to put the line-numbers
    numberstyle=\footnotesize,      % the size of the fonts that are used for the line-numbers
    backgroundcolor=\color{white},  % choose the background color. You must add \usepackage{color}
    commentstyle=\color{green},
    stringstyle =\color{orange},
    xleftmargin=1.5em,
    xrightmargin=1em,
    frame=tb,                       % adds a frame around the code
    tabsize=2,                      % sets default tabsize to 2 spaces
    captionpos=b,                   % sets the caption-position to top or bottom
    breaklines=true,                % sets automatic line breaking
    breakatwhitespace=false         % sets if automatic breaks should only happen at whitespace
    literate =%
     {#}{\#}1
}
\renewcommand*{\lstlistingname}{Source Code}
\renewcommand*{\lstlistlistingname}{List of Source Codes}

%% Tikz for (complex) colorful drawings ---------------------------------------------------------
\usepackage{tikz}
\usetikzlibrary{positioning,mindmap,shapes,shapes.multipart,shadows,arrows,patterns,topaths}
\usepackage{pgfplots}
\pgfplotsset{width=7cm,compat=newest}

\definecolor{yellow}{HTML}{FFFCCC}
\definecolor{green}{HTML}{CFFFCC}
\definecolor{blue}{HTML}{CCE9FF}
\definecolor{darkblue}{HTML}{99b7EE}
\definecolor{red}{HTML}{FFA8A8}
\definecolor{orange}{HTML}{FFD28F}

\lstset
{
    breaklines=true,
    tabsize=3,
    showstringspaces=false
}

\lstdefinestyle{Common}
{
    extendedchars=true,
    language={[Visual]Basic},
    frame=single,
    %===========================================================
    framesep=3pt,%expand outward.
    framerule=0.4pt,%expand outward.
    xleftmargin=3.4pt,%make the frame fits in the text area. 
    xrightmargin=3.4pt,%make the frame fits in the text area.
    %=========================================================== 
    rulecolor=\color{Red}
}

\lstdefinestyle{A}
{
    style=Common,
    backgroundcolor=\color{Yellow!10},
    basicstyle=\scriptsize\color{Black}\ttfamily,
    keywordstyle=\color{Orange},
    identifierstyle=\color{Cyan},
    stringstyle=\color{Red},
    commentstyle=\color{Green}
}

\lstdefinestyle{B}
{
    style=Common,
    backgroundcolor=\color{Black},
    basicstyle=\scriptsize\color{White}\ttfamily,
    keywordstyle=\color{Orange},
    identifierstyle=\color{Cyan},
    stringstyle=\color{Red},
    commentstyle=\color{Green}
}
% Default fixed font does not support bold face
\DeclareFixedFont{\ttb}{T1}{txtt}{bx}{n}{12} % for bold
\DeclareFixedFont{\ttm}{T1}{txtt}{m}{n}{12}  % for normal

% Custom colors
\usepackage{color}
\definecolor{deepblue}{rgb}{0,0,0.5}
\definecolor{deepred}{rgb}{0.6,0,0}
\definecolor{deepgreen}{rgb}{0,0.5,0}

% Python style for highlighting
\newcommand\pythonstyle{\lstset{
language=Python,
basicstyle=\ttm,
morekeywords={self},              % Add keywords here
keywordstyle=\ttb\color{deepblue},
emph={MyClass,__init__},          % Custom highlighting
emphstyle=\ttb\color{deepred},    % Custom highlighting style
stringstyle=\color{deepgreen},
frame=tb,                         % Any extra options here
showstringspaces=false
}}


% Python environment
\lstnewenvironment{python}[1][]
{
\pythonstyle
\lstset{#1}
}
{}