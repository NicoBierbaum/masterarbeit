\chapter{Schlussfolgerungen}
\section{Zusammenfassung der Ergebnisse}

In dieser Arbeit werden die Ergebnisse der Untersuchungen zu Dichteeigenschaften und Schrumpfverhalten des 316L-Edelstahls durch FDM-Druck präsentiert und diskutiert. Die Dichte der gedruckten Proben wurde in den verschiedenen Prozessphasen analysiert, wobei eine durchschnittliche Dichte von 6,87 \(\text{g/cm}^3\) im Stahlteil erreicht wird. Die Schrumpfung des Volumens war, wie zu erwarten, besonders hoch im Sinterprozess.

Die Ergebnisse der Zugversuche zeigen Unterschiede zwischen den liegenden und stehenden Zugproben. Die stehenden Zugproben weisen gegenüber den liegend gedruckten, eine geringere Zugfestigkeit auf. Aufgrund von Schwierigkeiten bei den stehenden Zugproben konzentriert sich die Analyse auf die liegenden Proben. Die mechanischen Kennwerte, einschließlich Streckgrenze (R${\textsubscript{p0.2}}$), Zugfestigkeit (R${\textsubscript{m}}$) und Bruchdehnung (A$_{\textsubscript{25 mm}}$), werden ermittelt und mit Werkstoffdatenblattwerten für 1.4404 Edelstahl verglichen.
Die durchschnittliche Dichte wies eine Abweichung von -14\% im Vergleich zum Datenblatt auf. Ein Vergleich mit anderen Studien zeigte eine Abweichung von -10\%, wobei jedoch unterschiedliche Materialien und eine Druckdüse mit einem Durchmesser von 0,6mm verwendet werden. Insgesamt bieten diese Erkenntnisse Einblicke in die Druckqualität und ermöglichen mögliche Optimierungsansätze für zukünftige FDM-Druckprozesse mit 316L-Edelstahl. Zusätzlich befindet sich unter \autoref{Anleitungen} eine Anleitung für den Umgang mit verstopftem Hotend. Auch bietet die Anleitung einen Einstieg in den Druck mit Metallfilament. Inspiration ist die offizielle Anleitung aus \Autocite{Prusa}, doch sie ist optimiert auf das Drucken mit Metallfilament.

\section{Ausblick}

Um die Dichte zu verbessern, wurde in \autocite{Quarto.2021} eine Düse mit einem Durchmesser von 0,6 mm verwendet, was zu einer Dichte von knapp über 7,2 \(\text{g/cm}^3\) führte. Daher ist es sinnvoll weitere Versuche mit verschiedenen Düsendurchmessern anzustellen. Zusätzlich können Schrumpfungseffekte durch Messungen an größeren Würfeln analysiert werden. Dies ermöglicht Rückschlüsse auf das Verhalten des Polymers bei größeren Volumina. Interessant sind Vergleiche bei der Schrumpfung der Masse. Da es schwieriger ist, das Polymer aus dem Inneren von größeren Bauteilen zu entfernen.
Ebenfalls interessant ist es den Druck von stehenden Zugproben zu optimieren. 
Das Unternehmen \textit{PT+A} bietet verschiedene Materialien an. Es ist ebenfalls sinnvoll mit der Absprache mit diesem Unternehmen, auch Versuche mit verschiedenen Materialien durchzuführen.\\
Des Weiteren ist es ebenfalls interessant, wie das Verfahren des Metalldrucks in der Industrie integriert werden kann. Hier bedarf es an Forschung nach Möglichkeiten.
\let\cleardoublepage\clearpage