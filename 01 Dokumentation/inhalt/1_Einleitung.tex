\chapter{Einleitung}
\label{einleitung}

\section{Hintergrund und Motivation}
Additive Fertigung \textit{(engl. Additive Manufacturing (AM))} wurde in den 1990ern eingeführt und beschreibt eine bis dahin neuartige Technologie 3D-Objekte in einem Schicht-für-Schicht Verfahren zu fertigen. Im Gegensatz zu AM steht die subtraktive Fertigung, auch als konventionelle Fertigung bezeichnet. Bei der die gewünschten Bauteile durch Abtrag von Material (z.B. beim Fräsen oder Drehen) hergestellt werden. Durch die additive Fertigung ergibt sich eine hohe Flexibilität und Designfreiheit, dies ist insbesondere für Prototypenherstellung, als auch immer mehr in der Serienproduktion von Bedeutung \autocite{Prof.Dr.Ing.ChristianSeidel.2023}.
Zu Beginn ist diese Technologie hauptsächlich in der Prototypenphase eingesetzt, da sich die Materialienauswahl auf Polymere beschränkt. So gab es vor allem viele Kunststoffe zur Auswahl, aber keine Materialien, die einem Stahl gleichkommen.
Heute ist AM im Weiteren Sinne auch für die Herstellung von funktionellen Teilen eingesetzt.  Firmen aus den Bereichen Luft- und Raumfahrt, Automobilherstellung, Energie und Medizin sind interessiert an dem Einsatz der AM zur Herstellung verschiedener Bauteile.
AM bringt folgende Vorteile mit sich:
\Autocite{Osama2019}
\begin{itemize}
    \item Große Freiheit im Design der Bauteile, was eine individuelle Anpassung an die Produktion ermöglicht. Zudem ist die Erstellung komplexer Bauteile deutlich einfacher als mit konventionellen Methoden.
    \item Der Konstruktionsprozess und die Prototypenphase sind deutlich schneller.
\end{itemize}

\section{Zielsetzung der Arbeit}

Das Ziel dieser Wissenschaftlichen Arbeit besteht darin, die Dichteeigenschaften und das Schrumpfungsverhalten des 316L (Edelstahl, Werkstoffkennzeichnung: 1.4404) der Firma \textit{PT+A} zu evaluieren. 
Genauer werden die Maße und das Gewicht in allen drei Verarbeitungsstufen aufgenommen, um daraus die Dichte zu bilden. Dadurch wird deutlich wieviel Material durch entbinden und sintern in Verlust gerät.
Im Idealfall ist die Dichte des gesinterten Bauteils ähnlich zu dem von handelsüblichem 316L.
Zusätzlich werden Zugproben gedruckt, die Aufschluss über die mechanischen Kennwerte des Materials geben. Diese werden auch mit handelsüblichem 316L verglichen.
    
% \section{Aufbau der Arbeit}

% Nach dieser Einleitung werden die technischen Grundlagen aufgearbeitet. Im Anschluss wird im dritten Kapitel wird der praktische Teil dieser Arbeit behandelt, dies umfasst den 3D-Druck und die optimierung der Druckparameter. Im vierten Kapitel werden die Ergebnisse der Dichtemessung und des Zugversuches diskutiert und verglichen. Zuletzt gibt es einen Ausblick.

