\chapter{Ergebnisse und Diskussion}

Im Fokus dieses Kapitels stehen die Ergebnisse und Diskussionen im Kontext der Evaluierung von Dichteeigenschaften und Schrumpfungsverhalten des 316L-Edelstahls von \textit{PT+A} für den 3D-Druck mittels Fused Deposition Modeling (FDM). Im vorangegangenem Kapitel wird die Erstellung der Proben und die Wissenschaftlichen Grundlagen erläutert.

\section{Dichte- und Schwindungsauswertungen der gedruckten Proben}

In diesem Unterkapitel werden die gedruckten Würfel- und Zugproben hinsichtlich der Dichte untersucht. Diese Messung wird einmal im Grün-, sowie im Braun-, als auch im späteren Sinterteil durchgeführt. Die Ergebnisse dieser Auswertung ist in \autoref{Grünteilmaße} bis \autoref{Sinterteilmaße} dargestellt.
Aus diesen Werten lassen sich die Schwindungen der einzelnen Parameter nach jedem Prozess ableiten. Dargestellt sind diese Werte in \autoref{Schrumpfung}.

\begin{table}[h]
    \centering
    \caption{Schrumpfung der Würfelproben im Braun- und Sinterteil}
      \begin{tabular}{ccccccc}
      \toprule
      \textbf{Teil} & \multicolumn{1}{c}{\textbf{h}} & \multicolumn{1}{c}{\textbf{l1}} & \multicolumn{1}{c}{\textbf{l2}} & \multicolumn{1}{c}{\textbf{V}} & \multicolumn{1}{c}{\textbf{m}} & \multicolumn{1}{c}{\textbf{Dichte}} \\
        \textbf{Braunteil} & 0 & 2,9 & 2,9 & 5,8 & 4,8 & -0,8 \\
        \textbf{Sinterteil} & 13,7 & 15,4 & 15,8 & 38,6 & 8 & -50 \\
      \bottomrule
      \end{tabular}%
    \label{Schrumpfung}%
  \end{table}%
  \FloatBarrier

Auffällig ist die völlige Schrumpfung des Volumens, welche erst nach dem Sinterprozess anliegt. Die Massschwindung nach beiden Prozessen lässt sich mit der Verflüchtigung von dem Bindermaterial erklären. 
Von den liegenden Zugproben wurden auch Masse und Gewicht gemessen. Daher lassen sich auch hier das Schrumpfungsverhalten darstellen. Diese Werte sind in \autoref{} angegeben.


\begin{table}[h]
    \centering
    \caption{Schrumpfung der liegenden Zugproben im Sinterteil}
      \begin{tabular}{ccccccc}
      \toprule
      \textbf{Teil} & \multicolumn{1}{c}{\textbf{l1}} & \multicolumn{1}{c}{\textbf{l2}} & \multicolumn{1}{c}{\textbf{V}} & \multicolumn{1}{c}{\textbf{m}} & \multicolumn{1}{c}{\textbf{Dichte}} \\
        \textbf{Sinterteil} & 13,7 & 15,4 & 15,8 & 38,6 & 8 & -50 \\
      \bottomrule
      \end{tabular}%
    \label{Schrumpfung Zugproben}%
  \end{table}%
  \FloatBarrier

\section{Vergleich dieser Werte}