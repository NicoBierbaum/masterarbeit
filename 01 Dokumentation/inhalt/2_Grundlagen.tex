\chapter{Theoretische Grundlagen}

Die additive Fertigung hat in den letzten Jahren Fortschritte gemacht und ist zu einem Bestandteil der modernen Fertigungstechnologie geworden. Innerhalb dieses Prozesses werden Bauteile über CAD \texttt{(computer aided design)} entworfen. Über Verfahren der additiven Fertigung werden diese Bauteile dann gefertigt. FDM-Druck hat sich als eine vielversprechende Methode etabliert, um komplexe dreidimensionale Objekte schichtweise aufzubauen. Während FDM-Druckverfahren traditionell auf Polymermaterialien ausgerichtet sind, hat die Weiterentwicklung dieser Technologie nun den Weg für den Einsatz von Metallfilamenten eröffnet. In diesem Kapitel werden die theoretischen Grundlagen erläutert.

\subsection{Additive Fertigung und der FDM-Prozess}

Additive Fertigung wird als Verfahren bezeichnet, bei dem Materialien entweder durch Verschmelzung, Bindung oder Verfestigung von Materialien wie flüssigen Harzen und Pulvern gemischt werden. 

\autocite{Osama2019}

2.2 Materialien für den FDM-Prozess\\
2.3 316L-Edelstahl als Druckmaterial\\
2.4 Dichte- und Schwindungsbegriffe\\